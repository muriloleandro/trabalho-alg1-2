\documentclass{article}

\title{2° Trabalho de Algoritmos e Estruturas de Dados I}
\author{
    Murilo Leandro Garcia \\ 
    Glauco Fleury
}
\date{}

\begin{document}
\maketitle

\section{Introducao}
Neste projeto foi implementado um TAD de conjuntos na linguagem C. Seguindo as especificações do projeto,
foram criadas as funções básicas de uma ED: criação, deleção, inserção de elementos, remoção de elementos, 
e impressão. Além disso, as seguintes funções especificas de conjuntos: união e intersecção.

Foram utilizadas duas estruturas de dados para armazenamento dos elementos do conjunto:
Árvore AVL e Left-Leaning Red Black Tree, em decorrência de sua eficiência para inserção e remoção de elementos
$O(\log n)$.

\section{Árvore AVL}
\subsection{Inserção}
\subsection{Remoção}

\section{Left-Leaning Red Black Tree}
\subsection{Inserção}
\subsection{Remoção}

\section{Conjunto}
\subsection{União na AVL}
\subsection{União na LLRBT}
\subsection{Intersecção na AVL}
\subsection{Intersecção na LLRBT}

\end{document}